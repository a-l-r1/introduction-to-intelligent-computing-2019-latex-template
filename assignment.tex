\documentclass[
  b5paper, 
  10pt, 
  AutoFakeBold, 
  AutoFakeSlant
]{article}

\usepackage{afterpage}
\usepackage{calc}

\usepackage{fontspec}
\usepackage[
  fontset=windowsnew
]{ctex}
\newCJKfontfamily[kai]\kaiti{KaiTi}

\setmainfont{Times New Roman}

\usepackage{amsmath}
\usepackage{amsfonts}
\usepackage{amssymb}

\usepackage[
  hidelinks
]{hyperref}

\usepackage{geometry}
\geometry{
  b5paper, 
  includeheadfoot, 
  left=1.45cm, 
  right=1.45cm, 
  top=1cm, 
  bottom=2.2cm, 
  head=1.3cm, 
  foot=1cm, 
  headsep=0cm
}

\usepackage{fancyhdr}
\pagestyle{fancy}
\fancyhf{}
\renewcommand{\headrulewidth}{0.5pt}

\usepackage[
  absolute, 
  overlay
]{textpos}
\setlength{\TPHorizModule}{1cm}
\setlength{\TPVertModule}{1cm}

\usepackage{mdframed}
\newmdenv[
  leftmargin=0cm, 
  rightmargin=0cm, 
  skipabove=0.15cm, 
  skipbelow=0.15cm, 
  innerleftmargin=0.1cm, 
  innerrightmargin=0.1cm, 
  innertopmargin=0.1cm, 
  innerbottommargin=0.1cm, 
  leftline=false, 
  rightline=false, 
  topline=false, 
  bottomline=true, 
  linewidth=0.75pt
]{headingleft}

\newcommand{\pageheader}[0]{%
\begin{textblock}{11.47}(1.65, 1.018)\noindent%
\begin{headingleft}%
\begingroup%
\xeCJKsetup{xeCJKactive=false}%
\setmainfont{KaiTi}%
\fontsize{9bp}{10.8bp}\normalfont\noindent%
\theauthor{}:\theattitle{}
\endgroup%
\end{headingleft}%
\end{textblock}%
\begin{textblock}{2}(13.85, 1.018)\noindent%
\begin{flushright}\noindent%
\begingroup%
\xeCJKsetup{xeCJKactive=false}%
\setmainfont{Calibri}%
\fontsize{10.5bp}{12.6bp}\normalfont\noindent%
\thepage%
\endgroup%
\end{flushright}%
\end{textblock}%
}%
\newcommand{\pageheaderunified}[0]{%
\ifnum\value{page}>1\pageheader\fi%
}%

\usepackage{everypage}
\AddEverypageHook{\pageheaderunified}

\usepackage[
  style=mla, 
  backend=biber, 
  style=numeric-comp
]{biblatex}

\addbibresource{assignment.bib}

\defbibheading{bibliography}{%
  \vskip 14pt\noindent\textbf{References:}%
}%

\usepackage{enumitem}
\newcommand{\bibalignmentwrapper}[1]{\item \parbox[b]{0.7cm}{#1}}
\defbibenvironment{bibliography}%
{%
  \begin{enumerate}[
    nosep,
    topsep=0pt, 
    left= 0.23cm .. 0.7cm, 
    label={[}\arabic*{]}
  ]%
}%
{%
  \end{enumerate}%
}%
{\item}

\renewcommand{\bibsetup}{%
}%
\renewcommand{\bibfont}{%
  \zihao{6}%
  \setlength{\leftskip}{0cm}%
  \setlength{\parskip}{0cm}%
  \setlength{\baselineskip}{12.8pt}%
  \setlength{\lineskip}{0cm}%
  \setlength{\leftskip}{0.04cm}%
  \setlength{\bibitemsep}{0cm}%
  \setlength{\bibinitsep}{0cm}%
}%

\usepackage{titlesec}
\titleformat{\section}{\zihao{5}\setmainfont{Times New Roman Bold}\heiti}{\parbox[b]{0.63cm}{\thesection}}{0cm}{}
\titlespacing{\section}{0cm}{8pt}{8pt}
\titleformat{\subsection}{\zihao{-5}\setmainfont{Times New Roman Bold}\heiti}{\parbox[b]{0.63cm}{\thesubsection}}{0cm}{}
\titlespacing{\subsection}{0cm}{2.2416pt}{2.2416pt}
\titleformat{\subsubsection}{\zihao{-5}\setmainfont{Times New Roman Bold}\heiti}{\parbox[b]{1cm}{\thesubsubsection}}{0cm}{}
\titlespacing{\subsubsection}{0cm}{0cm}{0cm}

\makeatletter
\newcommand{\theauthor}[0]{\@author}
\newcommand{\theattitle}[0]{\@title}
\newcommand{\theemail}[0]{\relax}
\newcommand{\email}[1]{\renewcommand{\theemail}[0]{#1}}
\newcommand{\thetelephone}[0]{\relax}
\newcommand{\telephone}[1]{\renewcommand{\thetelephone}[0]{#1}}
\makeatother

% 在此处填写文章相关信息
\author{作者名字}
\title{标题}
\email{邮件地址}
\telephone{电话}

\usepackage{graphicx}
\usepackage{xcolor}
\definecolor{emailcolor}{RGB}{5, 99, 193}
\newcommand{\emailcolor}[1]{\textcolor{emailcolor}{#1}}
\usepackage{hanging}

\begin{document}

\newgeometry{
  includeheadfoot, 
  left=1.45cm, 
  right=1.45cm, 
  top=1cm, 
  bottom=2.2cm, 
  head=0cm, 
  foot=1cm, 
  headsep=0cm
}
\afterpage{\aftergroup\restoregeometry}

\thispagestyle{empty}

\setlength{\parskip}{0cm}
{%
\heiti%
\zihao{4}%
\centerline{\theattitle}%
\addcontentsline{toc}{section}{\theattitle}%
}%
%
{%
\xeCJKsetup{%
  xeCJKactive=false%
}%
\setmainfont{FangSong}%
\vspace*{8pt}%
\zihao{-4}\fangsong%
\centerline{\scalebox{0.667}[1]{\theauthor\,\textsuperscript{1}}}%
\vspace*{12pt}%
}%
%
{%
\noindent%
\xeCJKsetup{%
  CJKspace=true%
}%
\fontsize{8pt}{9.6pt}\selectfont%
\hangafter=1\setlength{\hangindent}{0.7em}%
\centerline{\textsuperscript{1}(北京航空航天大学\enspace{}计算机学院, 北京\enspace{}海淀)}

\centerline{通讯作者: \theauthor, E-mail: \href{mailto:\theemail}{\emailcolor{\underline{\smash{\theemail}}}}, Tel: \thetelephone}%

~

}%

% 在此处撰写摘要和关键词
\zihao{-5}\hangpara{0.22cm}{1}{\heiti{}摘 要:} {\kaiti{}我能吞下玻璃而不伤身体。我能吞下玻璃而不伤身体。我能吞下玻璃而不伤身体。我能吞下玻璃而不伤身体。我能吞下玻璃而不伤身体。我能吞下玻璃而不伤身体。我能吞下玻璃而不伤身体。}

\zihao{-5}\hangpara{4.29em}{1}{\heiti{}关键词:} {\kaiti{}智能计算、智慧城市}

\zihao{-5}
\setlength{\rightskip}{0.35pt}
\setlength{\parskip}{0.1pt}

% 在此处撰写文章内容
\section{绪论}

% 示例文本来自《海伯利安》
他们关掉了仅有的一盏灯,房间内部仅仅被外面天空中的热闪电脉冲所照亮。黑暗忽隐忽现,房间被涂上了五光十色的色彩。有时,黑暗会持续好几秒,直到下一阵炮火猛烈倾泻。

领事摸索着自己的旅行包,从中掏出一个奇怪的装置,那东西比通信志大,有着古怪的装饰,前面有一个液晶触显,看上去像是那些历史全息像里的东西。

“秘密超光发射器?”布劳恩·拉米亚干巴巴地问。

领事的笑容中毫无幽默感:“这是个古老的通信志。出现于大流亡时期。”他从腰袋中掏出一块标准的微碟,插了进去。“跟霍伊特神父一样,我也必须先讲述其他人的故事,这样你们才能懂得我的故事。”

“真是要命啊,”马丁·塞利纳斯冷笑道,“他妈的这堆人中,难道我是唯一一个能够直截了当讲故事的人吗?我要多长时间……”

领事的行动把他自己都吓坏了。他站起身,旋即转向塞利纳斯,抓住那矮男人的斗篷和衬衣前襟,把他猛地压在墙上,拎在包装箱上。领事膝盖顶着塞利纳斯的小腹,前臂擒着他的喉咙:“再废话,诗人,我就让你去见阎王。”

塞利纳斯开始挣扎,但是他感觉气管被压得更紧了,他瞥到领事的眼神,于是停止了挣扎。他的脸色惨白。

卡萨德上校静静地,几乎是轻轻地将两人分开。“不会有评论了。”他说。他摸着皮带上的死亡之杖。

马丁·塞利纳斯走到圈子的远侧,他仍在揉脖子,一声不吭地跌落在一只箱子上。领事大步走向门口,吸了好几口气,然后走回人群。他对着每个人,除了诗人,说道:“对不起。只是……我从没想过要把这个故事讲给别人听。”

外面的光线涌现出红色,然后是白色,紧接着是蓝光,之后褪变成近乎黑暗。

“我们都了解,”布劳恩·拉米亚轻轻说,“我们都跟你一样,有过这种感觉。”

领事摸摸下嘴唇,点点头,艰难地清了清嗓子。他走到古老通信志旁,坐了下来。“录音没有这个仪器那么古老。”他说,“录的时间大约是在五十标准年前。录音放完后,我还会继续讲下去。”他顿了顿,似乎还有什么东西要讲,然后他摇摇头,大拇指按了按古旧的触显。

没有视频。声音是一个年轻男子的。背景声中,可以听见微风吹过青草、拂过嫩枝的声音,远处是滚滚的海浪声。

外面,亮光发狂闪动,远方太空站的拍子在加速。领事紧张地等待着爆裂声和冲击声。但是没有。他闭上眼睛,和众人一起倾听。

\section{研究现状}

他们信任我。在我开诚布公的论说中,他们开始相信重新加入人类大家庭……加入环网有多么棒。他们坚持只能有一个城市对外来人开放。我微笑着表示同意。现在新耶路撒冷有六千万人口,而整个大陆只有一千万犹太土著居民,他们大部分的生活来源依靠这个加入环网的城市。还需要等十年。可能花不了那么久。

希伯伦向环网开放之后,我有一点消沉。我发现了酒精,这个伟大的东西能够让我远离闪回与嗑电。格列莎一直留在医院里和我在一起,直到我完全戒掉酒瘾。很奇怪,在这个犹太星球上的诊所竟然属于天主教。我还记得那天晚上大厅里教袍摩擦出的沙沙声。

我的消沉变得平静,并逐渐远离。我的职业生涯还没有被破坏。我以正式的领事身份将妻儿都带到了布雷西亚。

我们在那里扮演的角色是多么微妙啊!我们所走的路线又是多么诡计多端。在数十年间,卡萨德上校、技术内核的军队都一直袭扰着驱逐者游群的流亡之处。现在议会和人工智能顾问理事会这两大巨头作出决议,决定在偏地检验一下驱逐者的兵力,看看他们到底有多大能耐。于是他们选中了布雷西亚。我承认,在我抵达之前的数十载里,布雷西亚人都代理我们行使权力。他们的社会是古色古香令人愉悦的普鲁士风格,极端的军国主义,经济上骄傲自负,目中无人,极度恐外,到了群情激昂地要征募军队以扫除“驱逐者威胁”的地步。最开始,一些人租借了一批火炬舰船,以便靠近驱逐者。他们有等离子武器。也有密集探针,装载有特制的病毒。

我犯了点小小的计算失误,当驱逐者部落到达的时候,我还身处布雷西亚。出现了几个月的误差。那时候本该是由一个军政分析小组来接替我的位置。

不过没关系。反正霸主的意图已经达成。军部坚定而快速的部署力完全通过了检验,霸主的利益没有受到任何实质上的损害。格列莎死了,当然。在首轮轰炸中就死了。还有阿龙,我十岁的儿子。他一直和我在一起……到战争结束时也还活着……但后来却死了,一些军部傻瓜撒下的饵雷和爆破炸药距离首都白金敏寺的难民营太近了。

他死的时候我没在他身边。

布雷西亚战役之后我得到了擢升。我被给予了一项任务,它是历来任职领事的人所能被委任的任务里最富挑战,也是最为机密的:我成为了负责与驱逐者直接谈判的外交官。

最开始我传输到鲸逖中心,与悦石议员的委员会和一部分人工智能顾问展开漫长的会议。我见到了悦石本人。计划相当地复杂。最主要的一点是:我们必须挑唆驱逐者主动发起进攻,而激怒他们的关键就在于海伯利安这颗星球。

驱逐者在布雷西亚战役之前就一直在观察海伯利安。我们的情报机构显示,他们深深地迷上了光阴冢和伯劳。此前他们对承载着卡萨德上校的霸主医疗舰船的攻击和其他的几次攻击,都是属于计算错误;在医疗船只被错误地判定为军事神行舰之时,他们的舰船长惶恐不已。在驱逐者看来,更糟糕的是,他们作出决定让登陆飞船降落在光阴冢附近。于是乎该船的司令官展露了他们抵御时间潮汐的能力,他们的突击队员遭到了伯劳大幅度的杀戮。在那之后,飞船船长回到游群接受了处决。

但是我们的情报机构显示驱逐者的计算错误并不完全是彻底的失败。他们获得了关于伯劳的有价值的信息。而且他们对于海伯利安的着迷也逐渐加深。

悦石曾向我解释霸主计划要怎样利用那种痴迷。

计划的核心在于我务必得激怒驱逐者去攻击霸主,而攻击的焦点必须是海伯利安本身。我由此开始明白,最终的战役是为了处理环网的内部政务,而不是要拔除驱逐者这颗眼中钉。几个世纪以来,技术内核的各方力量都反对海伯利安加入霸主。悦石解释说这不再是为人类的利益着想了,武力兼并海伯利安——以保护环网本身作为幌子——将会允许内核中更多的进步人工智能联合会获取权力。这样一来,内核中权力平衡的转变就会让议会和环网受益,具体途径则没有完全向我解释。驱逐者这一不可能妥协的潜在威胁将会被完全清除。霸主辉煌的新时代即将开始。

悦石解释说我不需要自愿前往,使命将会充满危险——不管对我的职业,还是人生来说,都是如此,但我还是接受了。

霸主给我提供了一艘私人飞船。我只要求了一处修改:配上一台古老的斯坦威钢琴。

我依靠霍金驱动独自旅行了好几个月。接下来的好几个月里,我在驱逐者游群定期移民的地段漫游。最终我的船舰被探测到并被俘获。他们相信我是一个信使,也明了我是一个间谍。他们中有人主张杀我,有人反对,辩论了很久,最终留我一条生路。他们也为是否要和我谈判争辩了不少时候,最终决定要这么做。

我并不想描述在游群生活的美妙——他们零重力的球形城市和彗星农场、刺丛,他们的微型环轨森林和迁徙河流,聚会礼拜生活的千颜万色与精细纹理。完全可以说,我相信驱逐者已经完成了环网人类在过去的几千年中都没有完成的事情:进化。当我们还住在自己的衍生文化——旧地生活苍白的浮影之中时,驱逐者已经开发了文化的新维度,包括美学、伦理学、生物化学、艺术和其他必须改变、进化的东西,人类灵魂也终于得以充分反映。

野蛮人,这是我们给予他们的称呼,但是在同时我们又怯懦地紧抓住自己的环网不放,就像当年的西哥特人\footnote{西哥特人(Visigoths):原居罗马帝国东北部,四世纪下半叶,因受到来自中亚的匈奴人的威胁,开始向西迁徙。公元378年安德里诺堡战役,西哥特人打败了罗马帝国的军队,410年西哥特人洗劫罗马城,随后占领了高卢南部阿基坦地区,以图卢兹作为首都,建立了西哥特王国,其疆域包括卢瓦尔河以南的西南高卢和比利牛斯半岛的大片土地。在西哥特人统治下的阿基坦,罗马高卢贵族的地产大多未受损害,他们依然按罗马帝国时的方式生活。}蜷缩在罗马逝去的辉煌中,宣布自己是文明人一样。

十个标准月之内,我就把我最大的秘密告诉了他们,而他们也把自己的秘密告诉了我。我尽自己所能极为详尽地解释了悦石的人为他们制定了什么样的计划,要将他们灭绝人世。我告诉他们环网科学家们对光阴冢的异常知之甚少,也告诉他们技术内核对海伯利安难以名状的惧怕。我详细描述说如果他们不惧危险企图占领海伯利安,就等于中了圈套,军部会倾巢出动,来到海伯利安星系,将他们歼灭干净。我将我所知道的一切和盘托出,并再次等待着死亡。

他们并没有杀我,反而告诉了我一些事。他们给我看了拦截到的超光信息、密光记录,还有他们四个半世纪以前从旧地星系逃出来时带走的一些记录。他们给我看的东西骇人且简单。

\section{发展趋势}

三八年的天大之误并不是个错误。旧地的死亡是蓄意的,是技术内核的成员和他们在霸主羽翼未丰的政府中的人类同伴策划的阴谋。早在失控的黑洞“意外”被放入旧地心脏部位的几十年前,他们就已经详尽地策划了大流亡的全过程。

环网、全局、人类霸主政权——它们全都是在这个最为邪恶的弑父行为之上建立起来的。现在它们又被一项不动声色精心策划的弑兄政策维系,他们杀戮其余的所有物种,只要对方露出一丁点儿竞争者的苗头。而驱逐者,在星际间自由流浪的唯一人类部族,唯一不受技术内核控制的种群,便是灭绝名单上的下一号人物。

我回到环网。环网时间已经过去了三十年。梅伊娜·悦石当上了首席执行官。希莉的叛乱成为了富有浪漫色彩的传奇,成为了霸主历史上一个无足轻重的小脚注。

我拜见了悦石。我告诉了她驱逐者向我透露的很多消息,但不是全部。我告诉她,他们知道为海伯利安打响的任何战役都是圈套,但是不管怎么说,他们还是会前来。我告诉她,驱逐者想让我成为海伯利安的领事,这样当战争爆发之时,我就会成为双重间谍。

我没有告诉她,他们已经承诺要给我一项装置,能够打开光阴冢,让伯劳挣开枷锁。

首席执行官悦石和我谈了很久。军部情报特工和我谈论得更为持久,有些谈话甚至持续了好几个月。他们运用技术和药物来确认我说的是真话,确认我没有隐瞒任何信息。驱逐者也很擅长运用技术和药物。我说的的确是真话。我只是保留了一些消息没有说出来。

最后,我被任命前往海伯利安。悦石提出要把那颗星球提升到保护体的地位,同时让我担任大使。我拒绝了这两个提议,但是我希望能够保留自己的私人飞船。我是乘坐一艘定期往返的神行舰上任的,而我自己的飞船也在数周之后搭乘一艘来访的火炬舰船抵达。它被留在了一条中继轨道,我随时可以召唤它下来,驾着它离开。

独自一人在海伯利安之时,我等待着。多年过去。我准许我的助手掌管这颗偏地星球,而我自己在西塞罗酒吧花天酒地,等待着。

驱逐者通过私人超光信息和我联络,而我向领事馆告了三周的假,让飞船降落在草之海附近一处与世隔绝之地,然后驾着它与他们的侦察艇在欧特云附近汇合,接走他们的特工——一个名叫安迪尔的女人——和一个技术专家三人小组,降落在笼头山脉的北方,距离光阴冢仅数公里远。

驱逐者没有远距传输器。他们的生命都被花费在星际间的长征上,遥望着环网的生命高速掠过,像是以癫狂速度播放的平面或全息电影。他们为时间而痴迷。技术内核向霸主提供并继续维护远距传输器。人类科学家和科学小组完全搞不懂远距传输器是如何运作的。驱逐者试图搞清楚,却失败了。但是,他们虽然失败了,却理解了怎样操控时空。

他们弄明白了时间潮汐,也就是环绕墓群的逆熵场。他们不能够制造这种能场,但是可以保护自己不受它的侵害,并且——从理论上——摧毁它们。光阴冢和它们的内在物体将不再逆时间运动。墓群将会“打开”。伯劳将会挣脱它的套索,不再被困在墓群的附近。里面所有的一切都将被释放。

驱逐者相信光阴冢是来自未来的人造之物,而伯劳则是一种用以拯救的武器,正等待着合适的双手将它捕获操控。伯劳教会将这个怪物视作复仇天使;驱逐者将它看作一种人类设计的工具,穿越时间回到过去,从技术内核的魔爪下挽救人类。安迪尔和技术专家此次前来是要进行校正和试验工作。

“你们现在并不会利用它,是吧?”我问。我们正站在叫作狮身人面像的建筑的阴影之下。

“现在不会,”安迪尔说,“要等到侵略战争一触即发的时候。”

“但是你说过这项装置要过好几个月才能起作用,”我说,“才能让墓群打开。”

安迪尔点点头。她有双深绿色的眼珠,个子很高,我能够分辨出她拟肤束装上装有动力的外骨骼上的微小细纹。“或许要经过一年甚至更久,”她说,“这项装置会使逆熵场逐渐衰退。但是这项过程一旦触发就再不能撤销。我们现在不会激活它,除非十大理事会已经决定必须要侵略环网。”

\section{}

领事转身看着索尔·温特伯:“你在对瑞秋唱什么曲子呢?”

学者挤出一丝笑容,搔着他短短的胡子:“这曲子来自一部古老的平面电影。大流亡前的电影。见鬼,是一切之前。”

“唱给我们听听。”布劳恩·拉米亚说,她明白领事在做什么了,她的脸色也惨白惨白的。

温特伯开始唱,他的声音很微弱,起初几乎听不见。但是那曲子铿锵有力,而且奇怪的是,非常吸引人。霍伊特神父拿起巴拉莱卡琴,开始和着曲子弹奏,音符中充满了信心。

布劳恩·拉米亚大笑了起来。马丁·塞利纳斯满怀敬畏地说:“我的天,我以前小时候唱过这首歌。这歌可真是古老啊。”

“可谁是魔法师?\footnote{索尔唱的歌来自《绿野仙踪》中的《我们去见魔法师》。《绿野仙踪》又名《奥兹国历险记》。}”
卡萨德上校问,他的声音在他的头盔中闷声作响,很奇怪,此时此刻这倒显得有趣得紧。

“奥兹又是什么?”拉米亚问。

“到底是谁要去见魔法师?”领事问,他感觉到他内心的黑色恐慌消退了,虽然只是消退了很小的一点点。

索尔·温特伯顿了顿,打算回答他们的问题,把这个平面电影的情节跟大家讲讲,这电影已经化为尘土好几个世纪了。

“没关系,”布劳恩·拉米亚说,“你稍后可以跟我们说。快,再唱一遍。”

在他们身后,黑暗吞噬了群山,风暴向下扫荡,越过荒野,向他们奔腾而来。天空继续发出血红之光,但是现在,虽然其他地方依旧漆黑一片,但东方的地平线微微泛起了鱼肚白。死寂之城在他们左边发着光,就像岩石皓齿。

布劳恩·拉米亚再次领头。索尔·温特伯的歌声更为嘹亮了,瑞秋愉快地扭动着身子。雷纳·霍伊特“哗”的一声甩掉他的披风,以便更方便地弹奏巴拉莱卡琴。马丁·塞利纳斯拿起一只空瓶子,扔向远远的沙地里,他也开始一起唱,令人惊讶的是,他那低沉的声音既有力又好听,完全将风声压了下去。

费德曼·卡萨德拉起护目镜,扛起武器,也加入了合唱队。领事也开口歌唱,他想了想那荒谬的歌词,朗声大笑,再次唱了起来。

就在黑暗涌现的地方,他们的足迹也变宽了。领事走到右边,卡萨德跟他并排走着,索尔·温特伯卡到他俩之间,就这样,他们不再是一列纵队,六个人现在是在并肩前行。布劳恩·拉米亚握住塞利纳斯的手,另一边握住了索尔的手。

他们仍旧高声歌唱,不再回头,大步大步地向前进,一路向下,迈进了山谷。

\setlength{\parskip}{0cm}
\printbibliography

\end{document}
